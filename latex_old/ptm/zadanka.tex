\documentclass[13pt]{article}
\usepackage[utf8]{inputenc}
\usepackage{polski}
\usepackage{amsmath}%matma
\usepackage{graphicx}%zdjecia
\usepackage{siunitx}
\usepackage{a4wide}
\usepackage[htt]{hyphenat}
\usepackage[framemethod=TikZ]{mdframed}
\usepackage{lib}
\usepackage{listings}

\definebox{definition}{black!60}

\begin{document}
\begin{enumerate}
\item Przeczytaj:

\texttt{https://forbot.pl/forum/topic/5180-kurs-programowanie-mikrokontrolerow-avr-w-jezyku-assembler-czesc-4/}

\item Napisz szablon kodu assembler i zacznij od części programu która poprawnie ustawi ADC - po ludzku wrzucasz jedynki i 0 w te bity rejestrów ADMUX oraz ADCSRA tak żeby się włączyło z daną szybkością etc.

Przykład:
\fig{1}{code}{}

\item Napisz kod który przeniesie dane z bufora ADC (kieszonka gdzie znajdują się kolejne pomiary napięcia z odpowiedniej nóżki mikokontrolera) na dany port (8 nóżek - bajt) 

Przykład:



main: 					//program glowny

sbic ADCSRA, ADIF		//jeśli bit ADIF w rejestrze ADCSRA nie jest ustawiony(konwersja nie gotowa) to pomiń następną instrukcje

rcall load\_OCR			//wywołaj poprogram który załaduje wartość z przetwornika ADC do rejestru OCR0

rjmp main				//instrukcja zostanie wykonana gdy konwersja nie jest jeszcze gotowa ,tym samym pętla ta wykonuje się, gdy oczekujemy na konwersje



load\_OCR:				//podprogram ładujący wartość z rejestru ADCH do rejestru OCR0

in R16, ADCH			//załaduj wartośc z ADCH do R16

out OCR0, R16			//załaduj wartośc z R16 do OCR0

sbi ADCSRA, ADSC		//ustaw bit ADSC w rejestrze ADCSRA , powoduje to wyzwolenie kolejnej konwersji ADC

ret						//powrót do podprogramu main

\end{enumerate}
\end{document}

