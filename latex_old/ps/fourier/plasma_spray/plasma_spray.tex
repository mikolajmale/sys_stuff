\documentclass[13pt]{article}
\usepackage[utf8]{inputenc}
\usepackage{polski}
\usepackage{amsmath}%matma
\usepackage{graphicx}%zdjecia
\usepackage{siunitx}
\usepackage{a4wide}
\graphicspath{{./figures/}}
\usepackage[framemethod=TikZ]{mdframed}

\newcommand{\definebox}[2]{%
  \newcounter{#1}
  \newenvironment{#1}[1][]{%
    \stepcounter{#1}%
    \mdfsetup{%
        frametitle={%
            \tikz[baseline=(current bounding box.east),outer sep=0pt]
            \node[anchor=east,rectangle,fill=white]
            {\strut \MakeUppercase#1~\csname the#1\endcsname\ifstrempty{##1}{}{:~##1}};}}%
    \mdfsetup{innertopmargin=1pt,linecolor=#2,%
        linewidth=2pt,topline=true,
        frametitleaboveskip=\dimexpr-\ht\strutbox\relax,}%
    \begin{mdframed}[]\relax%
    }{\end{mdframed}}%
}

\newcommand{\fig}[3]{
\begin{figure}[!h]
	\centering
	\includegraphics[width=#1\textwidth]{#2}
	\caption{#3}
\end{figure}
}

\definebox{definition}{black!60}

\title{Nakładanie warstw niskociśnieniową metodą Cold Spray}
\author{Mikołaj Małecki 237339 K00-24b}
\begin{document}
	\maketitle
\section{Wprowadzenie}
\fig{0.8}{bef.jpg}{lol}


\begin{thebibliography}{9}
\bibitem{additive} 
Shuo Yin
\textit{Additive Manufacturing}. 
\\\texttt{https://www.sciencedirect.com/science/article/pii/S2214860417302993}


\bibitem{wiki} 
From Wikipedia, the free encyclopedia
\textit{Cold spraying}. 
\\\texttt{https://en.wikipedia.org/wiki/Cold\_spraying}

\bibitem{wikil} 
From Wikipedia, the free encyclopedia
\textit{Dysza de Lavala}. 
\\\texttt{https://pl.wikipedia.org/wiki/Dysza\_de\_Lavala}
\end{thebibliography}
\end{document}

