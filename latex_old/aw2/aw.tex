\documentclass[13pt]{article}
\usepackage[utf8]{inputenc}
\usepackage{polski}
\usepackage{amsmath}%matma
\usepackage{graphicx}%zdjecia
\usepackage{siunitx}
\usepackage{a4wide}
\usepackage{datetime}
\graphicspath{{./figures/}}
\usepackage[framemethod=TikZ]{mdframed}
\usepackage{lib}

\newdate{date}{22}{05}{2020}
\date{\displaydate{date}}

\title{Ustawianie i pomiar narzędzi w dobie przemysłu 4.0}
\author{Mikołaj Małecki 237339}
\begin{document}
	\maketitle

% ------------------------------------------------- BEGIN ------------------------------------------------- %	

\section{Wprowadzenie}
\textbf{Industry 4.0} jest to trend we współczesnym przemyśle (czwartej rewolucji przemysłowej) polegający na integracji znanych mechanizmów z nowo rozwijanymi działami IT. Takimi jak:

\begin{enumerate}
\item \textit{cloud computing}
\item \textit{big data}
\item \textit{narrow AI (machine/deep/neural learning)}
\item \textit{IOT}
\item \textit{computer vision}
\end{enumerate}

\fig{0.7}{4}{Kluczowe technologie Przemysłu 4.0 \cite{4}}

W połączeniu z obecną dostępnością i ceną komponentów elektronicznych pozwoliło to na wykorzystanie tanich standardowych elementów w niemożliwy wcześniej sposób (np: drukarki 3D, smart czujniki ładowane raz na kilka lat). Rozwój przemysłu 4.0 pozwolił także na wykorzystanie skomplikowanych systemów (jak np: roboty przemysłowe) w nowy nie znany wcześniej sposób (np: smart factories).

Zastosowanie najnowszych systemów IT w przemyśle pozwoliło na zmianę poglądu na wiele aspektów w przemyśle jak wytwarzanie addytywne zamiast substraktywnego, czy współpraca człowieka ze zrobotyzowanymi stanowiskami. 

\newpage
% --------------------

\fig{0.5}{ur}{Roboty współpracujące firmy Universal Robots \cite{ur}}

Jeśli natomiast chodzi o pozycjonowanie i pomiary narzędzi w stosunku do przemysłu 4.0 to głównymi ulepszeniami są:

\begin{itemize}
\item akwizycja danych
\item analiza danych
\item integracja wielu funcjonalności dla jednego narzędzia
\item możliwość monitoringu stanu narzędzie w czasie rzeczywistym
\item możliwość kooperacji robotów z ludźmi
\end{itemize}

Przykładów, które można by przytoczyć przy tym temacie jest wiele. Postaram się dlatego omówić jeden specyficzny przykład smart narzędzia do roztaczania w przemyśle automotive \cite{smart}.

\fig{0.5}{smart}{Budowa omawianego narzędzia \cite{smart}}

\newpage
% --------------------

\section{Omówienie przykładu inteligentnego narzędzia}
Pomysł wykonania takiego narzędzia zrodził się ze względu na potrzebę zrekompensowania opadania wytaczadła i skutków sił skrawania. Pomimo zwiększenia zwinności wytwórczej poprzez stosowanie maszyn cnc w obszarze automotive, problemem pozostaje potrzeba wyspecjalizowanych narzędzi do roztaczania otworów wałów korbowych.

\fig{0.5}{tool}{Struktura wytaczarek liniowych: (a) konwencjonalne narzędzie do wytaczania linii ze wspornikiem i wieloma wkładkami skrawającymi, (b) projekt  systemu wytaczania z wykorzystaniem inteligentnego narzędzia. \cite{smart}}

Użycie takiego typu narzędzia w procesie technologicznym pozwala na:
\begin{itemize}
\item zautomatyzowane zmienianie narzędzi
\item wyeliminowanie podpory łożyskowej
\item bezpośredni pomiar parametrów procesu
\end{itemize}

Narzędzie składa się z [Rysunek 3]
\begin{enumerate}
\item systemu pomiarowego
\item kontrolera cyfrowego
\item płytki skrawającej
\item mechanizmu posuwu końcówki narzędzia
\item zasilania i systemu komunikacji
\end{enumerate}

\newpage

Pomairy sił skrawających opierają się na estymacji błedu w czasie rzeczywistym, poniżej porównanie pomiaru z użyciem estymatora do pomiaru dynamometrem.

\fig{0.5}{force}{Porównanie sił tnących dla klasycznego narzędzia  i omawianego inteligentnego narzędzia \cite{smart}}

\section{Podsumowanie}
Żeby w pełni zautomatyzować procesy technologiczne wytwarzania elementów potrzeba maszyn zwinnych, inteligentnych - podejmujących decyzje. W innym wypadku nie można mówić o pełnej automatyzacji. Dlatego jeśli chodzi o narzędzia w przemyśle 4.0 muszą one umieć się komunikować, zmieniać się w zależności od potrzeby i dostosowywać do istniejących warunków a nawet wysyłać informacje diagnostyczne do chmury.

% ------------------------------------------------- BIB ------------------------------------------------- %

\begin{thebibliography}{9}
\bibitem{4} 
dr inż. Małgorzata Kaliczyńska
\textit{Kluczowe technologie Przemysłu 4.0}. 
\\\texttt{https://automatykaonline.pl/Artykuly/Przemysl-4.0/Kluczowe-technologie-Przemyslu-4.0}

\bibitem{ur} 
Zbigniew Piątek
\textit{Coboty – roboty dostępne dla każdego}. 
\\\texttt{https://przemysl-40.pl/index.php/2017/11/05/roboty-dostepne-dla-kazdego/}

\bibitem{smart} 
Byung-Kwon Min, George O'Neal, Yoram Koren, Zbigniew Pasek
\textit{A smart boring tool for process control}. 
\\\texttt{https://www.sciencedirect.com/science/article/pii/S095741580200020X}
\end{thebibliography}
\end{document}


%\fig{0.7}{img}{... \cite{source}}

%\begin{definition}[Definition]
%...
%\end{definition}
