\documentclass[13pt]{article}
\usepackage[utf8]{inputenc}
\usepackage{polski}
\usepackage{amsmath}%matma
\usepackage{graphicx}%zdjecia
\usepackage{siunitx}
\usepackage{a4wide}
\usepackage{xurl}
\usepackage{datetime}
\graphicspath{{./figures/}}
\usepackage[framemethod=TikZ]{mdframed}
\usepackage{lib}


\newdate{date}{22}{05}{2020}
\date{\displaydate{date}}

\title{Narzędzia modułowe i mechatroniczne}
\author{Mikołaj Małecki, Adam Kamiński, Patryk Augustyniak}
\begin{document}
	\maketitle

% ------------------------------------------------- BEGIN ------------------------------------------------- %	

\section{Wprowadzenie}
Modułowe systemy narzędziowe umożliwiają zbudowanie zoptymalizowanego i dostosowanego zespołu do konkretnego zastosowania przy użyciu standardowych elementów. Stosunkowo niewielki zapas może stworzyć ogromną liczbę kombinacji, umożliwiając stosowanie wspólnych systemów narzędziowych w całym centrum wytwórczym.

Modułowy interfejs systemowy jest interfejsem pośrednim umieszczonym między interfejsem maszyny a chwytem lub wkładką. Aplikacje i maszyny stawiają różne wymagania modułowemu interfejsowi systemowemu.

\newpage
% --------------------





\newpage
% --------------------
\section{Przykłady}
\subsection{Narzędzie do roztaczania}
\fig{0.5}{smart}{Budowa omawianego narzędzia \cite{smart}}

Pomysł wykonania takiego narzędzia zrodził się ze względu na potrzebę zrekompensowania opadania wytaczadła i skutków sił skrawania. Pomimo zwiększenia zwinności wytwórczej poprzez stosowanie maszyn cnc w obszarze automotive, problemem pozostaje potrzeba wyspecjalizowanych narzędzi do roztaczania otworów wałów korbowych.\\


Użycie takiego typu narzędzia w procesie technologicznym pozwala na:
\begin{itemize}
\item zautomatyzowane zmienianie narzędzi
\item wyeliminowanie podpory łożyskowej
\item bezpośredni pomiar parametrów procesu
\end{itemize}

\fig{0.5}{tool}{Struktura wytaczarek liniowych: (a) konwencjonalne narzędzie do wytaczania linii ze wspornikiem i wieloma wkładkami skrawającymi, (b) projekt  systemu wytaczania z wykorzystaniem inteligentnego narzędzia. \cite{smart}}




Narzędzie składa się z [Rysunek 3]
\begin{enumerate}
\item systemu pomiarowego
\item kontrolera cyfrowego
\item płytki skrawającej
\item mechanizmu posuwu końcówki narzędzia
\item zasilania i systemu komunikacji
\end{enumerate}

\newpage

Pomairy sił skrawających opierają się na estymacji błedu w czasie rzeczywistym, poniżej porównanie pomiaru z użyciem estymatora do pomiaru dynamometrem.

\fig{0.5}{force}{Porównanie sił tnących dla klasycznego narzędzia  i omawianego inteligentnego narzędzia \cite{smart}}


\newpage

\subsection{Mechatroniczny nóż tokarski}
Usprawnienie tego narzędzia, zostało wprowadzone poprzez wprowadzenie własnego napędu oraz sensora położenia, całość sterowana jest za pomocą systemu komputerowego. Dzięki użytemu rozwiązaniu, na konwencjonalnej tokarce można uzyskać elementy o złożonym zarysie. \cite{ksiazka}

\begin{enumerate}


\item \textbf{Napęd}

Ruch liniowy realizowany za pomocą silnika krokowego (silnik krokowy użyty ze względu na dokładność posuwu) oraz śrubie tocznej. Ruch realizowany jest tylko w jednej osi (posuwowy), dzięki prowadnicom, które ograniczają swobodę ruchu narzędzia. Przy śrubie o skoku 2mm, minimalny krok systemu wynosi 0,01mm.
\item \textbf{Sensor położenia}

Do realizacji pomiaru położenia mogą być użyte rożne rozwiązania:
- sterowanie czasowe – przy zadanej prędkości obrotowej, położenie jest jest funkcja czasu i posuwu na obrót
\begin{itemize}
\item licznik obrotów wrzeciona – realizowane poprzez zliczanie impulsów wrzeciona z uwzględnieniem nastawionego posuwu
\item liniał pomiarowy skojarzony z ruchem imaka - klasyczny pomiar np. suwmiarka cyfrowa
\item optyczny system pomiarowy z zastosowaniem myszy optycznej
\end{itemize}

\item \textbf{System sterowania}

System sterowania to połączenie sterownika i komputera (lub jego uproszczonej wersji mikrokontrolera). W odpowiednim programie generowany jest cały przebieg obróbki:
\begin{itemize}
\item wizualizacja modelu (system CAD)
\item ustawienie parametrów obróbki
\item wygenerowanie programu obróbki
\item symulacja obróbki z uwzględnieniem wprowadzonych przez użytkownika danych
\item odczyt sygnałów z sensorów
\item sterowanie procesem
\end{itemize}

\end{enumerate}


	Wynikiem wprowadzenia mechatronicznego narzędzia do konwencjonalnej obrabiarki jest możliwość realizowania zadań przeznaczonych dla obrabiarek sterowanych numerycznie, przez obrabiarki konwencjonalne

\fig{0.5}{tok}{Narzędzie mechatroniczne rozszerzające możliwości kinematyczne obrabiarki \cite{lab}}

\newpage

\subsection{Narzędzie mechatroniczne do precyzyjnego toczenia wałków}
Kolejne rozwiązanie mechatroniczne w znacznym stopniu poprawia precyzje tokarki. 

\begin{itemize}
\item \textbf{Aktuator}

Realizacją sterowania zajmuje się tutaj aktuator piezoelektryczny. Jest precyzyjny, oraz dobrze dostosowany do szybkiego reagowania na sygnały.
\item \textbf{Sensor}

Role sensora spełnia tutaj laserowy czujnik przemieszczenia. Jest to wysoce precyzyjny czujnik. Układ działa na zasadzie emitera i detektora. Wiązka laserowa odbija się od lustra umieszczonego na narzędziu, wraca (po mierzalnym czasie) do detektora. Przy znanej prędkości lasera w otoczeniu, można obliczyć odległość emitera od narzędzia.
\item \textbf{System sterowania}

Za sterowanie, zbieranie danych, oraz przekazywanie ich do aktuatora odpowiada sterownik komputerowy. Obliczenia, oraz przekazywanie danych musi być szybkie, ponieważ od tego zależy precyzja narzędzia.
\end{itemize}

\fig{0.75}{narz}{Schemat narzędzia \cite{ksiazka}}


% ------------------------------------------------- BIB ------------------------------------------------- %
\newpage
\begin{thebibliography}{9}

\bibitem{smart} 
Byung-Kwon Min, George O'Neal, Yoram Koren, Zbigniew Pasek
\textit{A smart boring tool for process control}. 
\\\texttt{https://www.sciencedirect.com/science/article/pii/S095741580200020X}


\bibitem{lab} 
Hubert Skowronek
\textit{Narzędzia modułowe i mechatroniczne - instrukcja do labolatorium}.

\bibitem{ksiazka} 
P. Cichosz, M. Kuzinovski
\textit{Sterowane i mechatroniczne narzędzia skrawające}.

\bibitem{san} 
SANDVIK
\textit{Modularne rozwiązania}.

\url{https://www.sandvik.coromant.com/en-gb/knowledge/machine-tooling-solutions/tooling-considerations/pages/modular-solutions.aspx}
\end{thebibliography}
\end{document}


%\fig{0.7}{img}{... \cite{source}}

%\begin{definition}[Definition]
%...
%\end{definition}
